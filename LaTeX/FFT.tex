\documentclass{article}[a4paper, oneside, 11pt]
\synctex=1
\input{preamble}
\input{math}

\geometry{a4paper, left=25mm, right=25mm, top=25mm, bottom=25mm}
\title{Esercizio facoltativo* sulla FFT}
\author{B.~Tomelleri \thanksdf}

\begin{document}
\maketitle

%================================================================
%                            Riassunto
%================================================================
\begin{mdframed}
\textbf{Riassunto:} --- Si \`e studiato il comportamento di un diodo in silicio
PN, ricostruendone sperimentalmente la curva caratteristica $V-I$, al fine di
mettere in risalto la sua componente resistiva. A questo scopo \`e stata
esplorata un'ampia zona della curva, in particolar modo per alti valori di
$I$, dove questa caratteristica \`e maggiormente apprezzabile. Proponiamo dunque
un'estensione della legge di Shockley, tramite l'aggiunta di un termine
resistivo, in grado di descrivere l'elemento -ohmico- della giunzione.
La nuova curva caratteristica prevede un andamento sempre pi\`u lineare di $I$
al crescere di $V$ e questo risulta in accordo con l'andamento dei dati
sperimentali.
\\\\
PACS 01.40.-d – Education.\\
\end{mdframed}

%================================================================
%                         Introduzione
%================================================================
\section{Introduzione}
Si è rivisitato nel dominio delle frequenze lo studio di sistemi elettronici
e meccanici, finora analizzati solamente nel dominio del tempo, attraverso l'uso
di due strumenti fondamentali: la trasformata di Fourier Discreta (DFT) e la
detezione sincrona o Lock In Detection/Amplification (LIA).

%================================================================
%                         Cenni teorici
%================================================================
\subsection{Cenni Teorici}
La Trasformata di Fourier Discreta (o DFT) estende la trasformata di Fourier
analitica a sistemi con variabile dinamica discreta. In particolare
approfitteremo della velocità dell'algoritmo Cooley-Tukey \cite{FFT} o FFT
per la nostra analisi.

\subsubsection{Circuiti forzati e smorzati RLC}
In un circuito costituito da almeno una resistenza, un induttore ed un
condensatore (nel nostro caso collegati in serie) è possibile individuare una
frequenza caratteristica $f_0 = \frac{\omega_0}{2\pi}$, detta propria o di
risonanza, per cui il sistema è percorso da una corrente elettrica oscillante
nel tempo.
Infatti, quando il sistema è perturbato da una tensione alla frequenza
$f_0 := \frac{1}{2\pi \sqrt{LC}}$ 
le impedenze del condensatore e dell'induttore si annullano a vicenda, dunque
l'impedenza del circuito si trova al proprio valor minimo, cioè la sola
componente resistiva $R$ rimasta. Possiamo quindi descrivere il trasporto di
carica nel circuito con l'equazione di un oscillatore smorzato:
\begin{equation}\label{eq: RLC}
	\frac{\partial^2 Q}{\partial t^2} + \frac{R}{L}\frac{\partial Q}{\partial t}
	+ \omega_0^2 Q = 0   
\end{equation}
La cui soluzione, in termini della frequenza di oscillazione smorzata o
pseudo-frequenza angolare $\omega$, dello pseudo periodo $T$ e del tempo
caratteristico di smorzamento $\tau$:
\begin{align*}
	\omega &= \sqrt{\omega_0^2 - \frac{1}{\tau^2}} =
	\sqrt{\frac{1}{LC} - \frac{1}{\tau^2}} \\
	T &= \frac{2\pi}{\omega} \\
	\tau &= \frac{2L}{R}
\end{align*}
si può scrivere come:
\begin{align}\label{aln: Q(t)}
	Q(t) &= A e^{-t/\tau} \cos{(\omega t + \phi)} \\
	A &= \sqrt{c_1 c_2} \\
	\tan{\phi} &= j\frac{c_1 + c_2}{c_1 - c_2}
\end{align}
dove i coefficienti $c_1$ e $c_2$ dipendono dalle condizioni iniziali del
sistema. Secondo il nostro modello, la d.d.p. sulle armature del condensatore
è determinata dalla relazione costitutiva di $C$ e le condizioni iniziali
sono fissate dalla carica presente sulle armature $Q_0$ e dall'intensità
di corrente $I_0$ che circola nel circuito all'inizio dell'oscillazione:
\begin{align}\label{aln: Vc(t)}
	V_C(t) &= \frac{Q(t)}{C} = \frac{A}{C}e^{-\frac{t}{\tau}}\cos(\omega t + \phi)\\
	Q(t = 0) &= A\cos{\phi} := Q_0 \implies A = Q_0 \sqrt{1 + \tan^2{\phi}} \\
	I(t) &= \frac{\partial Q}{\partial t} = A e^{-t/\tau} \left[
	\frac{\cos{(\omega t + \phi)}}{\tau} + \omega\sin{(\omega t + \phi)} \right] \\
	I(t = 0) &= A \left[ \frac{\cos{\phi}}{\tau} + \omega\sin{\phi} \right] 
	\implies \phi = \arctan{\left( \frac{1}{\omega}\left[
	\frac{I_0}{Q_0} - \frac{1}{\tau}\right] \right) }
\end{align}

\subsubsection{Quality Factor}
Per descrivere la dissipazione media di energia da parte di sistemi oscillanti
nel tempo si è introdotta la quantità adimensionale \emph{Quality factor}:
\begin{equation}
	Q_f := 2\pi \frac{E_{\text{stored}}}{E_{\text{lost/cycle}}}
\end{equation}

Per un circuito RLC il trasferimento reciproco di energia elettrica e
magnetica, immagazzinate nel condensatore e nell'induttore rispettivamente,
è massimo sotto l'effetto di una forzante alla stessa $f_0$ di risonanza.
Supponendo che il circuito sia perturbato da un segnale sinusoidale
monocromatico, il metodo simbolico ci permette di legare la larghezza di riga
della risposta in frequenza del circuito al tempo caratteristico di
smorzamento, dunque al fattore di qualità dell'oscillazione.
La larghezza (di riga) a metà altezza
\begin{equation}
	\Delta \omega_{\text{FWHM}}	
\end{equation}
%================================================================
%                Metodo e apparato sperimentale
%================================================================
\section{Metodo e apparato sperimentale}
Non si è monitorata la temperatura dei componenti dei circuiti studiati,
tutti i collegamenti tra i componenti sono stati realizzati con cavi
terminanti in connettori a banana. L'uso di \verb+Arduino+\cite{arduino}
come sistema di acquisizione dati non permette di apprezzare le perturbazioni
dovute alla temperatura o ai collegamenti dei componenti nelle nostre
condizioni di lavoro.
%================================
%           Apparato
%================================
\subsection{Apparato}
L'apparato sperimentale consiste di diversi circuiti elettrici,
realizzati con componenti pre-assemblati in laboratorio. Per
monitorare la risposta dei circuiti si utilizzano i canali di un oscilloscopio
analogico ($50$ MHz) e uno digitale ($200$ MHz), mentre per l'acquisizione dei
segnali di d.d.p. compresi tra $0$ e $5$ V si fa uso del convertitore ADC
del MCU \verb+Arduino UNO+.

\subsubsection{Circuiti RLC}
Nel nostro caso l'induttore è costituito da due avvolgimenti concentrici
e coassiali, ciascuno dotato di 1500 spire, che montati in serie hanno un
fattore di auto-induzione $L \sim 0.5 \; \si{\henry}$ e i condensatori hanno
capacità dai valori nominali: $C = \left\{0.1,\; 0.22,\; 0.47\right\} \pm
10\% \; \si{\micro\farad}$. La componente resistiva $R$ del circuito è data
dalle resistenze interne del generatore di tensione ($r_G = 50 \; \si{\ohm}$
nominali) e dell'induttore, che indichiamo con: $r \approx 40 \;\si{\ohm}$.
Le $3000$ spire totali di filo di rame negli avvolgimenti infatti
influiscono apprezzabilmente e in maniera non banale sul trasferimento
di energia all'interno del circuito.
\begin{table}[H]
\begin{center}
\begin{tabular}{lll}
	\toprule
	$R1$ nom. [$\si{\ohm}$] & $R1$ mis. [$\si{\ohm}$] \\ 
	\midrule
	\midrule
	$0.22 \pm 3 \% $         	& $0.226 \pm 0.008$ \\
	$2.2 \pm 5 \% $          	& $2.212 \pm 0.008$ \\
	$22 \pm 5 \% $           	& $21.86 \pm 0.010$ \\ 
	$220 \pm 5 \% $          	& $216.22 \pm 0.07$ \\
	$2.2\; \si{k} \pm 5 \% $       & $2202.1 \pm 0.4$ \\
	$22\; \si{k} \pm 5 \% $       & ($21.7 \pm 0.3)10^3$ \\
	$0.22\; \si{M} \pm 5 \% $      & ($217 \pm 3)10^3$ \\
	\bottomrule
\end{tabular}
\caption{I valori delle resistenze poste in serie al diodo, riportate in
	valore nominale e misurate con multimetro digitale. \label{tab: res}}
\end{center}
\end{table}

%================================================================
%                   Analisi dati e Risultati
%================================================================
\section{Analisi dati e Risultati}
Per poter condurre un'analisi sui dati raccolti \`e stato innanzitutto
necessario convertire le letture digitalizzate nelle corrispondenti
grandezze fisiche, le coppie tensione-corrente relative al diodo.
Inizialmente si sono convertite le acquisizioni e le incertezze associate in
d.d.p. tramite i fattori di conversione per entrambi gli ADC,
determinati come descritto nel paragrafo \ref{sec: cal}.
Dunque, dalla caduta di tensione ai capi della resistenza $R1$ possiamo
determinare la corrente di lavoro del diodo grazie alla legge di Ohm. 
Si \`e quindi effettuato un filtraggio volto all'eliminazione degli outliers e 
dei punti meno significativi, assumendoli quali variabili indipendenti e di 
natura gaussiana. Per una discussione dettagliata si rimanda all'\nameref{app: 
A}. E’ opportuno sottolineare che, all’interno della stessa appendice, 
$\sigma_x^2$ rappresenta la varianza delle letture e non le incertezze ad esse
associate.

\subsection{Oscillatore a reazione con BJT}

Si riportano i valori ottimali dei parametri stimati dal fit e le relative
covarianze: 
\begin{align*}
R_{\text{diodo}} &= 46.112\iffalse 18 \fi \pm 0.007\iffalse 2 \fi \; \si{\mohm} 
&\sigma_{I_0, \eta V_T} &= 0.97  \\
\eta V_T &= 47.579\iffalse 86 \fi \pm 0.003\iffalse 3 \fi \; \si{\mV} 	
&\sigma_{I_0, R_d} &= -0.62 \\
I_0 &= 4.518\iffalse 0 \fi \pm 0.004\iffalse 3 \fi \; \si{\nA}
&\sigma_{I_0, \text{ofst}} &= -0.49 \\
\text{offset} &=\; - 2.204\iffalse 3 \fi \pm 0.007 \iffalse 3 \fi \; \si{\uA}
&\sigma_{\eta V_T, R_d} &= -0.70  \\ 
\chi^2/\text{ndof} &= 72207/251066
&\sigma_{\eta V_T, \text{ofst}} &= -0.43\\
\text{abs\_sigma} &= \rm False
&\sigma_{R_d, \text{ofst}} &= 0.22
\end{align*}
Infine si mostrano i dati acquisiti con sovrapposta la funzione di best-fit
nei grafici \ref{fig: sck_lin}, in scala lineare, e \ref{fig: sck_log}
in scala semilogaritmica. 

Per gli script si rimanda alla 
\href{https://github.com/LucaCiucci/relaz_seme/tree/master/Cartella_fit}
{cartella}, dove \verb+run.py+ esegue la corretta sequenza e \verb+config.py+
definisce i parametri fondamentali.

\begin{figure}[!ht]
\centering
	\includegraphics{phs1_41.pdf}
\label{fig: phs1_41}
\end{figure}
Si noti come, alla fine del segnale oscillante\footnote{alla fine del
semiperiodo positivo dell'onda quadra in uscita dal generatore di funzioni}
si riesce ad apprezzare un picco positivo di circa $100$ digit $\approx 97$ mV,
questo coincide esattamente con il fronte di salita dell'onda quadra in
ingresso al circuito RLC.
Questo si deve alla somma di due effetti: Quando l'onda passa da LOW a
HIGH, la corrente che scorre nella maglia RLC passa da circa $-100$ a
$100 \si{\micro\A}$. Dunque la bobina da $L=0.5 \si{\H}$ fornisce una tensione
$DV(t) = L \frac{\ud I}{\ud t}$ secondo la legge degli induttori.
Un secondo possibile contributo al picco di tensione alla
fine del segnale oscillante è dovuto all'accoppiamento capacitivo/capacità
parassita del diodo, che possiamo modellare come $Cjo = 1-4 \si{\pico\farad}$
in parallelo al diodo. Quando il fronte d'onda sale rapidamente, parte dell'onda
quadra viene lasciata passare dal condensatore, aggiungendo qualche $\si{\nano\V}$
al picco finale.
\begin{figure}[H]
	\centering 
		\includegraphics[width=16cm, height= 11cm]
		{100skip_linear}
	\caption{Dati acquisiti e funzione di best-fit \eqref{eq: model}. E' 
	stato rappresentato un punto ogni 100 per comodit\`a di visualizzazione.
	\label{fig: sck_lin}}
\end{figure}

\begin{figure}[!htp]
	\centering 
		\includegraphics[width=16cm, height= 9.9cm]{10skip_semilog}
	\caption{Dati acquisiti e funzione di best fit \eqref{eq: model} in 
	scala semilogaritmica. A scopo illustrativo sono stati rappresentati anche
	i dati della serie $220\si{\kohm}$. E' stato disegnato un punto ogni 10
	per comodit\`a di visualizzazione. \label{fig: sck_log}}
\end{figure}

%================================================================
%                          Conclusioni
%================================================================
\section{Conclusioni}

%================================================================
%                        Appendice B
%================================================================
\section{Appendice A: Filtraggio Dati}\label{app: A}
%================================
%         Introduzione
%================================
\subsection{Introduzione}
All'interno dell'acquisizione \`e stata raccolta un'ingente quantit\`a di dati,
suddivisibili in base alla resistenza scelta e dunque facenti riferimento
a zone differenti della curva. A seguito della calibrazione, ci si \`e quindi
posto il problema di effettuare l'eliminazione degli outliers in modo
indipendente dalla scelta del modello per il fit. Le serie effettuate
variando la resistenza, inoltre, si sovrappongono in alcune zone del
grafico. Dunque \`e stato necessario eliminare i dati che, non aggiungendo
informazioni utili, andavano a "sporcare" il grafico. Il sistema di filtraggio
di dati implementato nell'eseguibile si compone di 2 fasi:
la prima consiste nell'eliminazione degli outliers, la seconda dei dati non
significativi.
%================================
%         Procedimento
%================================
\subsection{Procedimento}
Supponiamo di avere una serie di dati $(x, y)$ e assumiamo che siano
indipendenti tra loro. Quest'ipotesi non \`e vera in generale, ma
\`e tanto pi\`u lecita quanto pi\`u la correlazione tra le varianze delle
misure su $x$ e $y$ \`e indipendente dai valori assunti dalle $x$ e $y$ stesse
e quanto pi\`u sono numerosi i dati racchiusi entro una deviazione standard
lungo $x$ per ciascun elemento: in questo caso, infatti, la correlazione
viene inclusa nella varianza lungo $y$.
Supponiamo inoltre che siano note a priori le $\sigma_x ^2 \coloneqq \var{x}$
e che la loro distribuzione di probabilit\`a sia normale (le distribuzioni
delle componenti sono approssimativamente gaussiane per il convertitore
di \verb+Teensy+, perlomeno utilizzando la risoluzione a 12 bit) secondo una 
matrice
di covarianza diagonale nella base $\left\{x, y\right\}$.
In ogni modo, i nostri dati $x$ e $y$ risultano indipendenti e
approssimativamente normali. Dunque le assunzioni risultano giustificate. 
Conseguentemente la densit\`a di probabilit\`a che un punto misurato in $x$ si trovi
a tale ascissa $x_i$, si ricava integrando lungo $y$ a $x$ fissata:
\[
	\ud P = \frac{1}{\sigma_{x_i} \sqrt{2\pi}}
	e^{-\frac{1}{2}{\frac{(x - x_i)^2}{\sigma_{x_i}^2}}} \ud x
.\] 
Dunque, ripetendo pi\`u volte la stessa misura, si otterr\`a la probabilit\`a:
\[
	P\left(\mid x - x_i \mid \leq \frac{\eps}{2} \right) = \eps G_{x_i} 
.\]
dove \[
	G_{x_i} \coloneqq \frac{1}{\sigma_{x_i} \sqrt{2\pi}}
	e^{-\frac{1}{2}{\frac{(x - x_i)^2}{\sigma_{x_i}^2}}}
.\] 
e $\eps > 0$ e $\eps \longrightarrow 0$. Scegliendo allora solo quelle misure $x$ per
cui vale $\mid x - x_i \mid \leq \frac{\eps}{2}$, queste saranno in numero
tendente a:
\[
	N_i \coloneqq N_{\text{tot}} \frac{G_{x_i}}{\sum_j G_{x_j}} =
		N_{\text{tot}} w_i
.\] 
che definisce implicitamente i pesi $w_i$ con cui si mediano le distribuzioni
di probabilit\`a gaussiane $G_{x_i}$.
Allora, posto:
\[
	G_{y_i} \coloneqq \frac{1}{\sigma_{y_i} \sqrt{2\pi}}
	e^{-\frac{1}{2}{\frac{(\mu_y - y_i)^2}{\sigma_{y_i}^2}}}
.\] 
Per il principio di massima verosimiglianza siamo quindi interessati a
massimizzare la quantit\`a:
\[
	\like = \prod_{i=1}^{n} \prod_{j=1}^{N_i} G_{y_i} = 
	\prod_{i=1}^{n} G_{y_i}^{N_i}
.\] 
Per la monotonia del logaritmo il problema equivale a massimizzare: 
\[
	\ln{\like} = \sum_{i=1}^{n}\ln{G_{y_i}}^{N_{\text{tot}}w_i} = 
	\frac{N_{\text{tot}}} {\sum_{j=1}^{n} G_{x_j}} 
	\sum_{i=1}^{n} G_{x_i} \ln{G_{y_i}}
.\] 
Per cui, a meno di costanti risulta:
\begin{equation}\label{eq: likeconst}
	\ln{\like} - \text{const.} \propto \sum_{i=1}^{n} -G_{x_i} \ln{\sigma_y}
	- \frac{1}{2} G_{x_i} \left( \frac{y_i - \mu_y}{\sigma_y} \right)^2
\end{equation}
Imponendo la condizione di stazionariet\`a rispetto a $\mu_y$ si ottiene dunque:
\begin{equation}\label{eq: muy}
	\mu_y = \sum_{i=1}^{n} y_i w_i 
\end{equation} 
Una volta sostituito in \eqref{eq: likeconst} quanto appena trovato per $\mu_y$
e imponendo la stessa condizione di stazionariet\`a rispetto a $\sigma_y$ si ha:
\begin{equation}\label{eq: sigmay}
	\sigma_y^2 = \sum_{i=1}^{n} (y_i - \mu_y)^2 w_i
\end{equation}
Infine \`e possibile ricavare la varianza di $\mu_y$ dalla definizione di
valore di aspettazione, riconducendola pi\`u volte a integrali di gaussiane
di altezze e ampiezze diverse:
\begin{align} \label{aln: varmuy}
	\var{\mu_y} &= \sum_{i=1}^{n} w_i^2 \sigma_y^2 + 
	\left(\frac{y_i}{\sum_{j=1}^{n} G_{x_j}} \right)^2 \frac{
	e^{-\frac{(x-x_i)^2}{3 \sigma_{x_i}^2}} +  
	\sqrt{3} \left( e^{-\frac{(x-x_i)^2}{\sigma_{x_i}^2}} -
	\sqrt{2} e^{-3 \frac{(x-x_i)^2}{4 \sigma_{x_i}^2}} \right)
	} {2 \sqrt{3}\pi \sigma_{x_i}^2} 
\end{align}
Riassumendo:\\
Nella \eqref{eq: muy} prendiamo una media dei campionamenti intorno ad un'
ascissa $x$ in esame, pesata sulla distanza che gli $x_i$ hanno da questa; 
intuitivamente lo interpretiamo come se stessimo applicando un 
\emph{blur a kernel gaussiano} ai punti acquisiti.
Effettivamente quello che stiamo facendo non \`e molto diverso da KDE monovariante,
dove per\`o scaliamo secondo il valore delle $y$.
Lo stesso ragionamento vale per $\sigma_y^2$, si ha una stima della varianza
dei dati la variare di $y$, pesata sulla distanza dai valori studiati. Dunque
$\mu_y \pm \sigma_y$ ci d\`a una descrizione della distribuzione dei nostri dati.

%================================
%           Var(muy)
%================================
\subsection{$\var{\mu_y}$}
Mentre $\sigma_y$ rappresenta la distribuzione dei dati intorno al valor medio 
$\mu_y$, $\var{\mu_y}$ ci indica l’incertezza sulla miglior stima di $y$.
Questo \`e utile per determinare la convergenza della stima in funzione dei
dati acquisiti. Infatti tanto pi\`u \`e elevata la densit\`a dei dati,
rispetto alla deviazione standard $\sigma_x$, tanto pi\`u la stima del valore
centrale risulta precisa.
Graficamente la banda di confidenza \`e pi\`u ristretta dove si concentrano
i dati.
Viceversa, la stessa tende ad allargarsi dove i dati sono sparsi, i.e. a
distanze paragonabili a $\sigma_x$. Numericamente, si vede dalla seconda somma
nell'espressione \eqref{aln: varmuy} che la stima del valore centrale \`e
statisticamente significativa solo quando si media su un intervallo campionato
con almeno qualche punto ogni deviazione $\sigma_x$: altrimenti $\sigma_y \to 0$
indicando cos\`i assenza di dati, mentre $\var{\mu_y}$ tende a $+\infty$ come
$\sim e^{x^2}$, indice della stessa insufficienza di dati al fine di stabilire
con precisione significativa il valore di $\mu_y$.
\begin{figure}[!htb]
	\centering 
 		\includegraphics[scale=0.55]{./varmuy.png}
 	\caption{La media $\mu_y$ \`e rappresentata dalla linea blu, mentre
	l'area in rosso indica il valore di $\var{\mu_y}$ al variare dei
	dati (in nero) lungo $x$. \label{fig: varmuy}}
\end{figure}
Nel caso opposto, in cui i dati sono "densi" (in confronto alle $\sigma_x$)
la seconda somma, per quanto computazionalmente intensiva, numericamente
sembrerebbe piccola in confronto alla prima: in realt\`a non lo \`e, ma
soprattutto questa non pu\`o essere trascurata, poich\'e \`e proprio la
quantit\`a che descrive la dipendenza dalla densit\`a stessa e dunque la
caratteristica convergenza/divergenza della precisione sulla stima centrale
fornita.

%================================
%        Filtro outliers
%================================
\subsection{Filtro outliers}
La parte pi\`u semplice nel filtraggio dati consiste nello scartare tutti quei
punti che distano da $\mu_y$ pi\`u di una soglia arbitraria $k$ di deviazioni
standard $\sigma_y$ (nel nostro caso \`e stato scelto $k = 2$, non critico,
trovato dopo una serie di prove). A differenza del classico metodo basato
sulla distanza dalla curva/modello di best fit, per il nostro criterio essa
\`e ininfluente. Questo risulta particolarmente utile in simili situazioni di
verifica del modello in quanto una selezione basata su un preliminare fit
risulterebbe influenzata dalla scelta della funzione in questione e
eliminerebbe tutti i dati che non risultano compatibili con essa.

%================================
%  Filtro dati non significativi
%================================
\subsection{Filtro dati non significativi}
Supponiamo di avere 2 set di dati fatti con diverse resistenze, il primo $(A)$
con una resistenza bassa, il secondo $(B)$ con una alta: Il primo set
esplorer\`a la regione ad alta corrente, mentre il secondo la regione di basse
correnti.
In generale i dati del primo si sovrapporranno anche nelle zone basse esplorate
dal secondo, per\`o senza aggiungere sostanziali informazioni rispetto a quanto
farebbe il secondo.
Esponiamo dunque il criterio sviluppato per ridurre l'influenza di questi
punti meno significativi sulla ricerca dei parametri di best-fit e sulla
rappresentazione finale dei dati.\\
Per capire se in un certo punto i dati di $A$ sono significativi, calcoliamo
la misura di significativit\`a che abbiamo sviluppato in \eqref{aln: varmuy}$:
\var{\mu_y}$ di $A$ e di $B$. Perci\`o se $\var{\mu_y}$ di $A$ \`e maggiore di 
$q\var{\mu_y}$ di $B$, con $q$ arbitrario (nell'esperienza \`e stato scelto
$q = 3$), questo indica che i dati di $A$
ci stanno dando "poca" informazione rispetto a quelli di $B$. A questo punto
\`e sufficiente controllare tutti i punti scorrendo su tutte le combinazioni
di set per eliminare i dati non significativi, che rendono meno
immediata l'interpretazione del grafico. Questo \`e ben visibile in scala
logaritmica sulle $y$, dove i punti con grandi incertezze o varianze tendono
a disperdersi rapidamente.
L’algoritmo \`e computazionalmente intensivo e richiede una corretta gestione
della memoria per evitare bolle di allocazione. Dunque \`e stato implementato
in \verb'C++' per praticit\`a e richiamato all’interno degli script (per dettagli si 
rimanda ai 
\href{https://github.com/LucaCiucci/relaz_seme/tree/master/Cartella_fit/filter_src}{sorgenti}).
Nelle figure di esempio sono mostrati i dati selezionati dall’algoritmo
(in nero) ed i dati scartati (in arancio). 

\begin{figure}[!htbp]
\centering
\begin{subfigure}{.5\textwidth}
	\centering 
 		\includegraphics[scale=0.5]{./nofilter.png}
		\caption{\label{fig: nofilter}}	
\end{subfigure}%
\begin{subfigure}{.5\textwidth}
	\centering 
 		\includegraphics[scale=0.5]{./filtered.png}
 	\caption{\label{fig: filtered}}
\end{subfigure}
\caption{Grafici in scala semilogaritmica prima (\ref{fig: nofilter}) e dopo 
(\ref{fig: filtered}) del filtraggio dati. I dati scartati sono stati
evidenziati in arancio. Per praticit\`a \`e stato rappresentato un centesimo
dei dati raccolti}
\end{figure}
\`E infine mostrato il confronto dei grafici delle $\var{\mu_y}$
tra due set successivi.
\begin{figure}[!h]
	\centering 
 		\includegraphics[scale=0.65]{comparison.png}
 	\caption{Confronto dei grafici delle $\var{\mu_y}$ su due set di
	dati consecutivi. \label{fig: comparison}}
\end{figure}

\subsection{Nota sull'implementazione}
Per determinare i parametri ottimali e le rispettive covarianze si \`e
implementato in \verb+Python+ un algoritmo di fit basato sui minimi quadrati
mediante la funzione \emph{curve\_fit} della libreria \texttt{Scipy}\cite{scipy}
Per tutti i fit su campionamenti digitali di \verb+Teensy+ si \`e imposto
$\rm{abs\_sigma} =$ False, in quanto la sorgente principale d'incertezza
sulle misure risulta non statistica/non determinata.

%================================================================
%                            END
%================================================================
\medskip
\bibliographystyle{IEEEtrandoi}
\bibliography{refs}
\end{document}
