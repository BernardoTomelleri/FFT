\documentclass{article}[a4paper, oneside, 11pt]

\input{preamble}
\input{math}

\geometry{a4paper, left=25mm, right=25mm, top=25mm, bottom=25mm}
\title{Esercizio obbligatorio sulla FFT}
\author{B.~Tomelleri \thanksdf}

\begin{document}
\maketitle

%================================================================
%                            Riassunto
%================================================================
\begin{mdframed}
\textbf{Riassunto:} --- Si \`e studiato il comportamento di un diodo in silicio
PN, ricostruendone sperimentalmente la curva caratteristica $V-I$, al fine di
mettere in risalto la sua componente resistiva. A questo scopo \`e stata
esplorata un'ampia zona della curva, in particolar modo per alti valori di
$I$, dove questa caratteristica \`e maggiormente apprezzabile. Proponiamo dunque
un'estensione della legge di Shockley, tramite l'aggiunta di un termine
resistivo, in grado di descrivere l'elemento -ohmico- della giunzione.
La nuova curva caratteristica prevede un andamento sempre pi\`u lineare di $I$
al crescere di $V$ e questo risulta in accordo con l'andamento dei dati
sperimentali.
\\\\
PACS 01.40.-d – Education.\\
PACS 01.50.Pa – Laboratory experiments and apparatus.
\end{mdframed}


%================================================================
%                         Introduzione
%================================================================
\section{Introduzione}

%================================================================
%                         Cenni teorici
%================================================================
\subsection{Cenni Teorici}

%================================================================
%                Metodo e apparato sperimentale
%================================================================
\section{Metodo e apparato sperimentale}
Onde evitare sostanziali aumenti di temperatura dei componenti, oltre ad
eventuali danni a questi stessi, si imprimono correnti impulsate sul circuito. 
%e le conseguenti variazioni di resistenza nei componenti 
La durata degli impulsi \`e inferiore ai 200 $\si{\us}$, che corrisponde ad
un'energia impressa inferiore a 1 $\si{\milli\J}$ (dunque ad un aumento
della temperatura del semiconduttore inferiore a 0.5 $\si{\K}$).

%================================
%           Apparato
%================================
\subsection{Apparato}
L'apparato sperimentale \`e costituito da un circuito, realizzato su basetta 
sperimentale (\emph{breadboard}), il cui scopo \`e generare le correnti
impulsate appena descritte attraverso il diodo $D1$ e la resistenza $R1$.
La misura della differenza di potenziale agli estremi della $R_1$ permette
di stimare la corrente che circola sul diodo.
Per poter esplorare una larga banda di correnti di lavoro
si \`e variata opportunamente la resistenza in serie al diodo, scegliendo tra
le seguenti:
\begin{table}[H]
    \begin{center}
	\begin{tabular}{lll}
	    \toprule
		$R1$ nom. [$\si{\ohm}$] & $R1$ mis. [$\si{\ohm}$] \\ 
	    \midrule
	    \midrule
	    $0.22 \pm 3 \% $         	& $0.226 \pm 0.008$ \\
	    $2.2 \pm 5 \% $          	& $2.212 \pm 0.008$ \\
	    $22 \pm 5 \% $           	& $21.86 \pm 0.010$ \\ 
		$220 \pm 5 \% $          	& $216.22 \pm 0.07$ \\
		$2.2\; \si{k} \pm 5 \% $       & $2202.1 \pm 0.4$ \\
		$22\; \si{k} \pm 5 \% $       & ($21.7 \pm 0.3)10^3$ \\
		$0.22\; \si{M} \pm 5 \% $      & ($217 \pm 3)10^3$ \\
	    \bottomrule
	\end{tabular}
	\caption{I valori delle resistenze poste in serie al diodo, riportate in
		valore nominale e misurate con multimetro digitale. \label{tab: res}}
    \end{center}
\end{table}

Per una descrizione del metodo di misurazione delle resistenze si rimanda 
all'\nameref{app: A}.
D'altra parte, per fornire una tensione regolabile e sufficientemente stabile
durante un impulso, nel circuito si impiega il condensatore $C1$, la cui
tensione, carica e scarica sono controllate dalla maglia destra del circuito.
Nel circuito si distingue inoltre la maglia che si occupa di collegare, sotto
istruzione di un microcontrollore, il condensatore alla serie $R1-D1$
attraverso un MOS-FET.\\
Il circuito \`e alimentato da 2 tensioni continue, fornite da un alimentatore
stabilizzato switching e da un Buck Boost Converter.
%,alle maglie di "switch" e di "carica/scarica" rispettivamente
Un segnale fornito su $T2$ provoca la carica del condensatore mentre su $T3$
la sua scarica. Un segnale (invertito) su $T1$ innesca l'impulso di corrente
sul diodo.\\
La tensione di C$1$ \`e misurata attraverso un partitore di tensione collegato 
ad \verb+OUT3+ ed alla scheda di controllo, mentre la d.d.p. tra i punti
\verb+OUT1+ e \verb+OUT2+ \`e misurata direttamente.\\
Come MCU per la gestione dell'apparato si \`e scelta la scheda 
\verb+Teensy 3.2+\cite{teensy}. Questa si occupa del controllo dei segnali e 
delle letture analogiche. In particolare \verb+Teensy+ permette la lettura
analogica sincronizzata differenziale veloce, essendo dotato di due ADC,
entrambi con una risoluzione reale di 12 bit. La lettura differenziale \`e
fondamentale per acquisire coppie di dati per le tensioni ai capi del diodo
e della resistenza.

\begin{figure}[!htb]
	\centering 
 		\includegraphics[scale=0.5]{./measure.png}
 	\caption{Schema circuitale del sistema di lettura (verso\texttt{Teensy})
	\label{sch:rdng}}
\end{figure}

\begin{figure}[h!t]
	\centering 
 		\includegraphics[scale=0.5]{./gestione.png}
 	\caption{Circuito globale per la gestione del diodo \label{sch:gest}}
\end{figure}

%===============================
%         Calibrazione
%================================
\subsection{Calibrazione}\label{sec: cal}

La calibrazione dei canali \verb+ADC0+ e \verb+ADC1+ \`e stata effettuata
basandosi sulle letture differenziali, rispettivamente fra i pin \verb+A10+
e \verb+A11+; fra \verb+A12+ e \verb+A13+. E’ stata calibrata la lettura
di ciascuna delle due coppie separatamente.

I pin volti alla lettura della tensione maggiore sono stati collegati al 
centrale di un potenziometro, a sua volta collegato ad un generatore di 
potenziale con tensione in ingresso pari a $3.3\si{\V}$. Gli altri due pin
sono stati invece collegati a massa.

Variando la resistenza e, conseguentemente, la differenza di potenziale in 
ingresso, \`e stato possibile calcolare media, deviazione standard campione
e deviazione standard dalla media relative a ciascuna tensione per 
un gruppo di 10000 misure effettuate con \verb+Teensy+ (si veda il 
\href{https://github.com/LucaCiucci/relaz_seme/blob/master/sketches/teensy_calib/teensy_calib.ino}{relativo sketch}).
Il centrale del potenziometro e la massa sono state, inoltre, collegate ad un
multimetro digitale, in modo da poter associare la tensione misurata in
Volt alla corrispondente lettura digitalizzata dal MCU. Abbiamo assunto come
incertezze sulle misure i valori dichiarati nelle specifiche del multimetro,
mentre come errori sulle letture digitali abbiamo preso le deviazioni standard
dalla media.

Dopodich\'e le misure relative a ciascuna coppia di boccole sono state poste 
all’interno di un grafico con i dati raccolti da \verb+Teensy+ sulle ascisse
e i valori misurati col multimetro sulle ordinate. Dunque si \`e effettuato un
fit lineare attraverso la legge:
\begin{equation}
y(x; m, q)  = m x + q
\end{equation}

lasciando liberi entrambi i parametri $m$ e $q$. Riportiamo sotto le miglior
stime dei parametri determinati dai fit relativi ad \verb+ADC0+ ed \verb+ADC1+ 
e le rispettive covarianze:
\begin{align*}
	&\qquad \texttt{ADC0}	&&\qquad  \texttt{ADC1} \\
	m_0 &= 0.7968 \pm  0.0033  \; \mathrm{mV/digit} 
	& m_1  &= 0.7954 \pm 0.0026  \; \mathrm{mV/digit} \\
	q_0 &= - 0.18 \pm 0.34  \; \si{\mV} 	
	&q_1 &= 3.84  \pm 0.26  \; \si{\mV} \\
	\corr{m_0} {q_0} &= - 0.15      &\corr{m_1}{q_1} &= - 0.09 \\
	\chi^2/\text{ndof} &= 145/15	&\chi^2/\text{ndof} &= 96/15 \\ 
	\text{abs\_sigma} &= \rm False	&\text{abs\_sigma} &= \rm False
\end{align*}

\begin{figure}[H]
	\centering
	\begin{subfigure}{.5\textwidth}
		\def\svgwidth{\columnwidth}
		\includesvg{cal0}
	\label{fig: cal0}
	\end{subfigure}%
	\begin{subfigure}{.5\textwidth}
		\def\svgwidth{\columnwidth}
		\includesvg{cal1}
	\label{fig: cal1}
	\end{subfigure}
\end{figure}

I coefficienti cos\`i ricavati ci permetteranno di fornire una buona stima 
dei valori centrali relativi alla misura in Volt delle letture analogiche 
(digit). Il $\chi^2$ risulta essere sovrastimato in quanto il MCU non ci 
fornisce una risposta perfettamente lineare ai segnali in ingresso. Dalle 
specifiche di \verb+Teensy 3.2+, sappiamo che lo scostamento dall'andamento
lineare pu\`o essere quantificato col $7 \%$ della lettura in digit.\\
Per maggiori informazioni riguardo alle stime delle incertezze nelle 
conversioni si rimanda direttamente al 
\href{https://github.com/LucaCiucci/relaz_seme/blob/master/Cartella_fit/funzioni.py}
{relativo script}.

%================================
%       Acquisizione dati
%================================
\subsection{Acquisizione dati}
Per ciascun valore scelto della resistenza $R1$ si \`e acquisita una presa 
dati automatizzata tramite una 
\href{https://github.com/LucaCiucci/relaz_seme/blob/master/sketches/teensy_differenziale_definitivo/teensy_differenziale_definitivo.ino}
{routine} programmata in \verb+Teensy+: il condensatore viene caricato ad una
tensione prefissata; quando il MCU misura questo valore prestabilito,
invia il segnale su $T1$ che avvia l'acquisizione sincronizzata e si attende
un tempo 50 $\si{\us}$ per permettere al MOS-FET di entrare in conduzione
($R2$ \`e stata scelta di $1 \si{\kohm}$, dunque c'\`e un apprezzabile ritardo
tra segnale e impulso) e di scartare eventuali segnali spuri.
A questo punto si inizia a memorizzare una serie da 100 coppie di letture
-sincronizzate- (a meno di sfasamenti nell'ordine delle centinaia di
nanosecondi), si ferma l'impulso e si trasmettono i dati al computer.
Se la tensione al punto \verb+OUT2+ ha raggiunto valori maggiori di $3.3\si{\V}$
l'acquisizione si ferma per non danneggiare \verb+Teensy+, altrimenti si
ricomincia caricando il condensatore ad una tensione pi\`u alta.

%================================================================
%                   Analisi dati e Risultati
%================================================================
\section{Analisi dati e Risultati}
Per poter condurre un'analisi sui dati raccolti \`e stato innanzitutto
necessario convertire le letture digitalizzate nelle corrispondenti
grandezze fisiche, le coppie tensione-corrente relative al diodo.
Inizialmente si sono convertite le acquisizioni e le incertezze associate in
d.d.p. tramite i fattori di conversione per entrambi gli ADC,
determinati come descritto nel paragrafo \ref{sec: cal}.
Dunque, dalla caduta di tensione ai capi della resistenza $R1$ possiamo
determinare la corrente di lavoro del diodo grazie alla legge di Ohm. 
Si \`e quindi effettuato un filtraggio volto all'eliminazione degli outliers e 
dei punti meno significativi, assumendoli quali variabili indipendenti e di 
natura gaussiana. Per una discussione dettagliata si rimanda all'\nameref{app: 
A}. E’ opportuno sottolineare che, all’interno della stessa appendice, 
$\sigma_x^2$ rappresenta la varianza delle letture e non le incertezze ad esse
associate.


Risulta evidente, osservando i punti campionati con la resistenza da
$R1 = 220 \; \si{\kohm}$, la presenza di un rumore di natura non statistica
molto maggiore delle incertezze sulle misure. 
Questo si deve probabilmente all'influenza del circuito di lettura
stesso sulle piccole correnti in gioco, che avevamo supposto trascurabile
nel nostro modello. Non potendo dunque attribuire degli errori significativi
per questa presa dati, essa non \`e stata considerata nella procedura di
fit eseguita successivamente, in quanto i relativi punti avrebbero avuto un
peso sovrastimato in confronto agli altri, introducendo cos\`i un bias
nei risultati finali.
 
Si riportano i valori ottimali dei parametri stimati dal fit e le relative
covarianze: 
\begin{align*}
	R_{\text{diodo}} &= 46.112\iffalse 18 \fi \pm 0.007\iffalse 2 \fi \; \si{\mohm} 
	&\sigma_{I_0, \eta V_T} &= 0.97  \\
	\eta V_T &= 47.579\iffalse 86 \fi \pm 0.003\iffalse 3 \fi \; \si{\mV} 	
	&\sigma_{I_0, R_d} &= -0.62 \\
	I_0 &= 4.518\iffalse 0 \fi \pm 0.004\iffalse 3 \fi \; \si{\nA}
	&\sigma_{I_0, \text{ofst}} &= -0.49 \\
	\text{offset} &=\; - 2.204\iffalse 3 \fi \pm 0.007 \iffalse 3 \fi \; \si{\uA}
	&\sigma_{\eta V_T, R_d} &= -0.70  \\ 
	\chi^2/\text{ndof} &= 72207/251066
	&\sigma_{\eta V_T, \text{ofst}} &= -0.43\\
	\text{abs\_sigma} &= \rm False
	&\sigma_{R_d, \text{ofst}} &= 0.22
\end{align*}
Infine si mostrano i dati acquisiti con sovrapposta la funzione di best-fit
nei grafici \ref{fig: sck_lin}, in scala lineare, e \ref{fig: sck_log}
in scala semilogaritmica. 

Per gli script si rimanda alla 
\href{https://github.com/LucaCiucci/relaz_seme/tree/master/Cartella_fit}
{cartella}, dove \verb+run.py+ esegue la corretta sequenza e \verb+config.py+
definisce i parametri fondamentali.

\begin{figure}[H]
	\centering 
		\includegraphics[width=16cm, height= 11cm]
		{100skip_linear}
	\caption{Dati acquisiti e funzione di best-fit \eqref{eq: model}. E' 
	stato rappresentato un punto ogni 100 per comodit\`a di visualizzazione.
	\label{fig: sck_lin}}
\end{figure}

\begin{figure}[!htp]
	\centering 
		\includegraphics[width=16cm, height= 9.9cm]{10skip_semilog}
	\caption{Dati acquisiti e funzione di best fit \eqref{eq: model} in 
	scala semilogaritmica. A scopo illustrativo sono stati rappresentati anche
	i dati della serie $220\si{\kohm}$. E' stato disegnato un punto ogni 10
	per comodit\`a di visualizzazione. \label{fig: sck_log}}
\end{figure}

%================================================================
%                          Conclusioni
%================================================================
\section{Conclusioni}

%================================================================
%                        Appendice B
%================================================================
\section{Appendice A: Filtraggio Dati}\label{app: A}
%================================
%         Introduzione
%================================
\subsection{Introduzione}
All'interno dell'acquisizione \`e stata raccolta un'ingente quantit\`a di dati,
suddivisibili in base alla resistenza scelta e dunque facenti riferimento
a zone differenti della curva. A seguito della calibrazione, ci si \`e quindi
posto il problema di effettuare l'eliminazione degli outliers in modo
indipendente dalla scelta del modello per il fit. Le serie effettuate
variando la resistenza, inoltre, si sovrappongono in alcune zone del
grafico. Dunque \`e stato necessario eliminare i dati che, non aggiungendo
informazioni utili, andavano a "sporcare" il grafico. Il sistema di filtraggio
di dati implementato nell'eseguibile si compone di 2 fasi:
la prima consiste nell'eliminazione degli outliers, la seconda dei dati non
significativi.
%================================
%         Procedimento
%================================
\subsection{Procedimento}
Supponiamo di avere una serie di dati $(x, y)$ e assumiamo che siano
indipendenti tra loro. Quest'ipotesi non \`e vera in generale, ma
\`e tanto pi\`u lecita quanto pi\`u la correlazione tra le varianze delle
misure su $x$ e $y$ \`e indipendente dai valori assunti dalle $x$ e $y$ stesse
e quanto pi\`u sono numerosi i dati racchiusi entro una deviazione standard
lungo $x$ per ciascun elemento: in questo caso, infatti, la correlazione
viene inclusa nella varianza lungo $y$.
Supponiamo inoltre che siano note a priori le $\sigma_x ^2 \coloneqq \var{x}$
e che la loro distribuzione di probabilit\`a sia normale (le distribuzioni
delle componenti sono approssimativamente gaussiane per il convertitore
di \verb+Teensy+, perlomeno utilizzando la risoluzione a 12 bit) secondo una 
matrice
di covarianza diagonale nella base $\left\{x, y\right\}$.
In ogni modo, i nostri dati $x$ e $y$ risultano indipendenti e
approssimativamente normali. Dunque le assunzioni risultano giustificate. 
Conseguentemente la densit\`a di probabilit\`a che un punto misurato in $x$ si trovi
a tale ascissa $x_i$, si ricava integrando lungo $y$ a $x$ fissata:
\[
	\ud P = \frac{1}{\sigma_{x_i} \sqrt{2\pi}}
	e^{-\frac{1}{2}{\frac{(x - x_i)^2}{\sigma_{x_i}^2}}} \ud x
.\] 
Dunque, ripetendo pi\`u volte la stessa misura, si otterr\`a la probabilit\`a:
\[
	P\left(\mid x - x_i \mid \leq \frac{\eps}{2} \right) = \eps G_{x_i} 
.\]
dove \[
	G_{x_i} \coloneqq \frac{1}{\sigma_{x_i} \sqrt{2\pi}}
	e^{-\frac{1}{2}{\frac{(x - x_i)^2}{\sigma_{x_i}^2}}}
.\] 
e $\eps > 0$ e $\eps \longrightarrow 0$. Scegliendo allora solo quelle misure $x$ per
cui vale $\mid x - x_i \mid \leq \frac{\eps}{2}$, queste saranno in numero
tendente a:
\[
	N_i \coloneqq N_{\text{tot}} \frac{G_{x_i}}{\sum_j G_{x_j}} =
		N_{\text{tot}} w_i
.\] 
che definisce implicitamente i pesi $w_i$ con cui si mediano le distribuzioni
di probabilit\`a gaussiane $G_{x_i}$.
Allora, posto:
\[
	G_{y_i} \coloneqq \frac{1}{\sigma_{y_i} \sqrt{2\pi}}
	e^{-\frac{1}{2}{\frac{(\mu_y - y_i)^2}{\sigma_{y_i}^2}}}
.\] 
Per il principio di massima verosimiglianza siamo quindi interessati a
massimizzare la quantit\`a:
\[
	\like = \prod_{i=1}^{n} \prod_{j=1}^{N_i} G_{y_i} = 
	\prod_{i=1}^{n} G_{y_i}^{N_i}
.\] 
Per la monotonia del logaritmo il problema equivale a massimizzare: 
\[
	\ln{\like} = \sum_{i=1}^{n}\ln{G_{y_i}}^{N_{\text{tot}}w_i} = 
	\frac{N_{\text{tot}}} {\sum_{j=1}^{n} G_{x_j}} 
	\sum_{i=1}^{n} G_{x_i} \ln{G_{y_i}}
.\] 
Per cui, a meno di costanti risulta:
\begin{equation}\label{eq: likeconst}
	\ln{\like} - \text{const.} \propto \sum_{i=1}^{n} -G_{x_i} \ln{\sigma_y}
	- \frac{1}{2} G_{x_i} \left( \frac{y_i - \mu_y}{\sigma_y} \right)^2
\end{equation}
Imponendo la condizione di stazionariet\`a rispetto a $\mu_y$ si ottiene dunque:
\begin{equation}\label{eq: muy}
	\mu_y = \sum_{i=1}^{n} y_i w_i 
\end{equation} 
Una volta sostituito in \eqref{eq: likeconst} quanto appena trovato per $\mu_y$
e imponendo la stessa condizione di stazionariet\`a rispetto a $\sigma_y$ si ha:
\begin{equation}\label{eq: sigmay}
	\sigma_y^2 = \sum_{i=1}^{n} (y_i - \mu_y)^2 w_i
\end{equation}
Infine \`e possibile ricavare la varianza di $\mu_y$ dalla definizione di
valore di aspettazione, riconducendola pi\`u volte a integrali di gaussiane
di altezze e ampiezze diverse:
\begin{align} \label{aln: varmuy}
	\var{\mu_y} &= \sum_{i=1}^{n} w_i^2 \sigma_y^2 + 
	\left(\frac{y_i}{\sum_{j=1}^{n} G_{x_j}} \right)^2 \frac{
	e^{-\frac{(x-x_i)^2}{3 \sigma_{x_i}^2}} +  
	\sqrt{3} \left( e^{-\frac{(x-x_i)^2}{\sigma_{x_i}^2}} -
	\sqrt{2} e^{-3 \frac{(x-x_i)^2}{4 \sigma_{x_i}^2}} \right)
	} {2 \sqrt{3}\pi \sigma_{x_i}^2} 
\end{align}
Riassumendo:\\
Nella \eqref{eq: muy} prendiamo una media dei campionamenti intorno ad un'
ascissa $x$ in esame, pesata sulla distanza che gli $x_i$ hanno da questa; 
intuitivamente lo interpretiamo come se stessimo applicando un 
\emph{blur a kernel gaussiano} ai punti acquisiti.
Effettivamente quello che stiamo facendo non \`e molto diverso da KDE monovariante,
dove per\`o scaliamo secondo il valore delle $y$.
Lo stesso ragionamento vale per $\sigma_y^2$, si ha una stima della varianza
dei dati la variare di $y$, pesata sulla distanza dai valori studiati. Dunque
$\mu_y \pm \sigma_y$ ci d\`a una descrizione della distribuzione dei nostri dati.

%================================
%           Var(muy)
%================================
\subsection{$\var{\mu_y}$}
Mentre $\sigma_y$ rappresenta la distribuzione dei dati intorno al valor medio 
$\mu_y$, $\var{\mu_y}$ ci indica l’incertezza sulla miglior stima di $y$.
Questo \`e utile per determinare la convergenza della stima in funzione dei
dati acquisiti. Infatti tanto pi\`u \`e elevata la densit\`a dei dati,
rispetto alla deviazione standard $\sigma_x$, tanto pi\`u la stima del valore
centrale risulta precisa.
Graficamente la banda di confidenza \`e pi\`u ristretta dove si concentrano
i dati.
Viceversa, la stessa tende ad allargarsi dove i dati sono sparsi, i.e. a
distanze paragonabili a $\sigma_x$. Numericamente, si vede dalla seconda somma
nell'espressione \eqref{aln: varmuy} che la stima del valore centrale \`e
statisticamente significativa solo quando si media su un intervallo campionato
con almeno qualche punto ogni deviazione $\sigma_x$: altrimenti $\sigma_y \to 0$
indicando cos\`i assenza di dati, mentre $\var{\mu_y}$ tende a $+\infty$ come
$\sim e^{x^2}$, indice della stessa insufficienza di dati al fine di stabilire
con precisione significativa il valore di $\mu_y$.
\begin{figure}[!htb]
	\centering 
 		\includegraphics[scale=0.55]{./varmuy.png}
 	\caption{La media $\mu_y$ \`e rappresentata dalla linea blu, mentre
	l'area in rosso indica il valore di $\var{\mu_y}$ al variare dei
	dati (in nero) lungo $x$. \label{fig: varmuy}}
\end{figure}
Nel caso opposto, in cui i dati sono "densi" (in confronto alle $\sigma_x$)
la seconda somma, per quanto computazionalmente intensiva, numericamente
sembrerebbe piccola in confronto alla prima: in realt\`a non lo \`e, ma
soprattutto questa non pu\`o essere trascurata, poich\'e \`e proprio la
quantit\`a che descrive la dipendenza dalla densit\`a stessa e dunque la
caratteristica convergenza/divergenza della precisione sulla stima centrale
fornita.

%================================
%        Filtro outliers
%================================
\subsection{Filtro outliers}
La parte pi\`u semplice nel filtraggio dati consiste nello scartare tutti quei
punti che distano da $\mu_y$ pi\`u di una soglia arbitraria $k$ di deviazioni
standard $\sigma_y$ (nel nostro caso \`e stato scelto $k = 2$, non critico,
trovato dopo una serie di prove). A differenza del classico metodo basato
sulla distanza dalla curva/modello di best fit, per il nostro criterio essa
\`e ininfluente. Questo risulta particolarmente utile in simili situazioni di
verifica del modello in quanto una selezione basata su un preliminare fit
risulterebbe influenzata dalla scelta della funzione in questione e
eliminerebbe tutti i dati che non risultano compatibili con essa.

%================================
%  Filtro dati non significativi
%================================
\subsection{Filtro dati non significativi}
Supponiamo di avere 2 set di dati fatti con diverse resistenze, il primo $(A)$
con una resistenza bassa, il secondo $(B)$ con una alta: Il primo set
esplorer\`a la regione ad alta corrente, mentre il secondo la regione di basse
correnti.
In generale i dati del primo si sovrapporranno anche nelle zone basse esplorate
dal secondo, per\`o senza aggiungere sostanziali informazioni rispetto a quanto
farebbe il secondo.
Esponiamo dunque il criterio sviluppato per ridurre l'influenza di questi
punti meno significativi sulla ricerca dei parametri di best-fit e sulla
rappresentazione finale dei dati.\\
Per capire se in un certo punto i dati di $A$ sono significativi, calcoliamo
la misura di significativit\`a che abbiamo sviluppato in \eqref{aln: varmuy}$:
\var{\mu_y}$ di $A$ e di $B$. Perci\`o se $\var{\mu_y}$ di $A$ \`e maggiore di 
$q\var{\mu_y}$ di $B$, con $q$ arbitrario (nell'esperienza \`e stato scelto
$q = 3$), questo indica che i dati di $A$
ci stanno dando "poca" informazione rispetto a quelli di $B$. A questo punto
\`e sufficiente controllare tutti i punti scorrendo su tutte le combinazioni
di set per eliminare i dati non significativi, che rendono meno
immediata l'interpretazione del grafico. Questo \`e ben visibile in scala
logaritmica sulle $y$, dove i punti con grandi incertezze o varianze tendono
a disperdersi rapidamente.
L’algoritmo \`e computazionalmente intensivo e richiede una corretta gestione
della memoria per evitare bolle di allocazione. Dunque \`e stato implementato
in \verb'C++' per praticit\`a e richiamato all’interno degli script (per dettagli si 
rimanda ai 
\href{https://github.com/LucaCiucci/relaz_seme/tree/master/Cartella_fit/filter_src}{sorgenti}).
Nelle figure di esempio sono mostrati i dati selezionati dall’algoritmo
(in nero) ed i dati scartati (in arancio). 

\begin{figure}[!htbp]
\centering
\begin{subfigure}{.5\textwidth}
	\centering 
 		\includegraphics[scale=0.5]{./nofilter.png}
		\caption{\label{fig: nofilter}}	
\end{subfigure}%
\begin{subfigure}{.5\textwidth}
	\centering 
 		\includegraphics[scale=0.5]{./filtered.png}
 	\caption{\label{fig: filtered}}
\end{subfigure}
\caption{Grafici in scala semilogaritmica prima (\ref{fig: nofilter}) e dopo 
(\ref{fig: filtered}) del filtraggio dati. I dati scartati sono stati
evidenziati in arancio. Per praticit\`a \`e stato rappresentato un centesimo
dei dati raccolti}
\end{figure}
\`E infine mostrato il confronto dei grafici delle $\var{\mu_y}$
tra due set successivi.
\begin{figure}[!h]
	\centering 
 		\includegraphics[scale=0.65]{comparison.png}
 	\caption{Confronto dei grafici delle $\var{\mu_y}$ su due set di
	dati consecutivi. \label{fig: comparison}}
\end{figure}

\subsection{Nota sull'implementazione}
Per determinare i parametri ottimali e le rispettive covarianze si \`e
implementato in \verb+Python+ un algoritmo di fit basato sui minimi quadrati
mediante la funzione \emph{curve\_fit} della libreria \texttt{Scipy}\cite{scipy}
Per tutti i fit su campionamenti digitali di \verb+Teensy+ si \`e imposto
$\rm{abs\_sigma} =$ False, in quanto la sorgente principale d'incertezza
sulle misure risulta non statistica/non determinata.

%================================================================
%                            END
%================================================================
\medskip
\bibliographystyle{IEEEtrandoi}
\bibliography{refs}
\end{document}
