\documentclass{article}[a4paper, oneside, 11pt]
\synctex=1
\input{preamble}
\input{math}

\geometry{a4paper, left=25mm, right=25mm, top=25mm, bottom=25mm}
\title{Esercizio facoltativo* sulla FFT}
\author{B.~Tomelleri \thanksdf}

\begin{document}
\maketitle

%================================================================
%                         Introduzione
%================================================================
\section{Introduzione}
Si è rivisitato nel dominio delle frequenze lo studio di sistemi elettronici
e meccanici, finora analizzati solamente nel dominio del tempo, attraverso l'uso
di due strumenti fondamentali: la trasformata di Fourier Discreta (DFT) e la
detezione sincrona o Lock In Detection/Amplification (LIA).

%================================================================
%                         Cenni teorici
%================================================================
\section{Cenni Teorici}
La Trasformata di Fourier Discreta (o DFT) estende la trasformata di Fourier
analitica a sistemi con variabile dinamica discreta. In particolare
approfitteremo della velocità dell'algoritmo Cooley-Tukey \cite{FFT} o FFT
per la nostra analisi.

\subsection{Circuiti forzati e smorzati RLC}
In un circuito costituito da almeno una resistenza, un induttore ed un
condensatore (nel nostro caso collegati in serie) è possibile individuare una
frequenza caratteristica $f_0 = \frac{\omega_0}{2\pi}$, detta propria o di
risonanza, per cui il sistema è percorso da una corrente elettrica oscillante
nel tempo.
Infatti, quando il sistema è perturbato da una tensione alla frequenza
$f_0 := \frac{1}{2\pi \sqrt{LC}}$ 
le impedenze del condensatore e dell'induttore si annullano a vicenda, dunque
l'impedenza del circuito si trova al proprio valor minimo, cioè la sola
componente resistiva $R$ rimasta. Possiamo quindi descrivere il trasporto di
carica nel circuito con l'equazione di un oscillatore smorzato:
\begin{equation}\label{eq: RLC}
	\frac{\partial^2 Q}{\partial t^2} + \frac{R}{L}\frac{\partial Q}{\partial t}
	+ \omega_0^2 Q = 0   
\end{equation}
La cui soluzione, in termini della frequenza di oscillazione smorzata o
pseudo-frequenza angolare $\omega$, dello pseudo periodo $T$ e del tempo
caratteristico di smorzamento $\tau$:
\begin{align*}
	\omega &= \sqrt{\omega_0^2 - \frac{1}{\tau^2}} =
	\sqrt{\frac{1}{LC} - \frac{1}{\tau^2}} \\
	T &= \frac{2\pi}{\omega} \\
	\tau &= \frac{2L}{R}
\end{align*}
si può scrivere come:
\begin{align}\label{aln: Q(t)}
	Q(t) &= A e^{-t/\tau} \cos{(\omega t + \phi)} \\
	A &= \sqrt{c_1 c_2} \\
	\tan{\phi} &= j\frac{c_1 + c_2}{c_1 - c_2}
\end{align}
dove i coefficienti $c_1$ e $c_2$ dipendono dalle condizioni iniziali del
sistema. Secondo il nostro modello, la d.d.p. sulle armature del condensatore
è determinata dalla relazione costitutiva di $C$ e le condizioni iniziali
sono fissate dalla carica presente sulle armature $Q_0$ e dall'intensità
di corrente $I_0$ che circola nel circuito all'inizio dell'oscillazione:
\begin{align}\label{aln: Vc(t)}
	V_C(t) &= \frac{Q(t)}{C} = \frac{A}{C}e^{-\frac{t}{\tau}}\cos(\omega t + \phi)\\
	Q(t = 0) &= A\cos{\phi} := Q_0 \implies A = Q_0 \sqrt{1 + \tan^2{\phi}} \\
	I(t) &= \frac{\partial Q}{\partial t} = A e^{-t/\tau} \left[
	\frac{\cos{(\omega t + \phi)}}{\tau} + \omega\sin{(\omega t + \phi)} \right] \\
	I(t = 0) &= A \left[ \frac{\cos{\phi}}{\tau} + \omega\sin{\phi} \right] 
	\implies \phi = \arctan{\left( \frac{1}{\omega}\left[
	\frac{I_0}{Q_0} - \frac{1}{\tau}\right] \right) }
\end{align}

\subsection{Quality Factor}
Per descrivere la dissipazione media di energia da parte di sistemi oscillanti
nel tempo si è introdotta la quantità adimensionale \emph{Quality factor}:
\begin{equation}\label{eq: Qf}
	Q_f := 2\pi \frac{E_{\text{stored}}}{E_{\text{lost/cycle}}}
\end{equation}

Per un circuito RLC il trasferimento reciproco di energia elettrostatica e
magnetostatica, immagazzinate nel condensatore e nell'induttore rispettivamente,
è massimo sotto l'effetto di una forzante alla stessa $f_0$ di risonanza.
Possiamo esprimere l'energia interna all'induttore come $U_M = \frac{1}{2}LI^2$
e quella accumulata dal condensatore come $U_E = \frac{1}{2C}Q^2$.
Assumendo che all'inizio dell'oscillazione $t = 0 := t_0$ tutta l'energia del
circuito sia contenuta all'interno dell'induttore, possiamo identificare
$E_{\rm stored}$ con $U_M$ e caratterizzare l'energia dissipata in ogni
periodo sfruttando la relazione di Joule per gli effetti termico-dissipativi:
\begin{align}
	E_{\text{stored}} =& U_{M_0} = \frac{1}{2} LI_0^2 \\
	E_{\text{lost/cycle}} =& \langle P_{\rm Joule} \rangle_t \cdot T = 
	\frac{1}{2} RI_0^2 \cdot T\\
	R :=& r + r_G 
\end{align}
Supponendo che il circuito sia perturbato da un segnale sinusoidale
monocromatico, il metodo simbolico ci permette di legare la larghezza di riga
della risposta in frequenza del circuito al tempo caratteristico di
smorzamento, dunque al fattore di qualità dell'oscillazione.
Nell'approssimazione di oscillazione sottosmorzata $\omega_0 \gg \frac{1}{\tau}
\implies \omega = \sqrt{\omega_0^2 - \frac{1}{\tau^2}} \approx \omega_0$
possiamo approssimare la larghezza a metà altezza della curva di risonanza con 
\begin{equation}\label{eq: FWHM}
	\Delta \omega_{\text{FWHM}}	\approx \sqrt{3} RC \omega_0^2 = 
	\sqrt{3} \frac{R}{L} = \frac{2 \sqrt{3}}{\tau} 
\end{equation}
Per cui ci si aspetta che la larghezza della campana di risonanza sia
proporzionale alla severità dello smorzamento/perdita di energia
dell'oscillazione.
In altre parole, per un circuito RLC il fattore di qualità atteso è:
\begin{equation}\label{eq: QfFWHM}
	Q_f \approx 2\pi \frac{\frac{1}{2}L I_0^2}{\frac{1}{2}R I_0^2 \cdot T_0}
	= \omega_0 \frac{L}{R} = \omega_0 \frac{\tau}{2} 
	= \sqrt{3} \frac{\omega_0}{\Delta \omega_{\text{FWHM}}} 
\end{equation}

\subsection{Oscillatore a reazione con BJT}
Il secondo tipo di sistema oscillante studiato è un circuito dotato di
un amplificatore invertente, un transistor a giunzione bipolare (BJT) montato
ad emettitore comune (emitter-follower), ed un anello di feedback.
Secondo il criterio di stabilità di Barkhausen, affinché un circuito
elettrico lineare possa sostenere la propria auto-oscillazione è necessaria
la presenza di un feedback positivo, i.e. lo sfasamento totale dovuto
all'anello di feedback dev'essere un multiplo di $2\pi$. Nel nostro caso la
condizione è soddisfatta interponendo una rete di sfasamento, costituita da
3 filtri RC passa-alto passivi, che inverte nuovamente segno
$\Delta \phi = \pi$
al segnale in uscita dall'amplificatore. Vale la pena ricordare un
modo equivalente per ottenere un sistema auto-oscillante, in cui lo stesso
sfasamento è dato dalla serie di un induttore ed un condensatore,
quello che va sotto il nome di oscillatore di Colpitts.
%================================================================
%                Metodo e apparato sperimentale
%================================================================
\section{Metodo e apparato sperimentale}
Non si è monitorata la temperatura dei componenti dei circuiti studiati,
tutti i collegamenti tra i componenti sono stati realizzati con cavi
terminanti in connettori a banana. L'uso di \verb+Arduino+\cite{arduino}
come sistema di acquisizione dati non permette di apprezzare le perturbazioni
dovute alla temperatura o ai collegamenti dei componenti nelle nostre
condizioni di lavoro.
%================================
%           Apparato
%================================
\subsection{Apparato}
L'apparato sperimentale consiste di diversi circuiti elettrici,
realizzati con componenti pre-assemblati in laboratorio. Per
monitorare la risposta dei circuiti si utilizzano i canali di un oscilloscopio
analogico ($50$ MHz) e uno digitale ($200$ MHz), mentre per l'acquisizione dei
segnali di d.d.p. compresi tra $0$ e $5$ V si fa uso del convertitore ADC
del MCU \verb+Arduino UNO+.

\subsubsection{Circuiti RLC}
Nel nostro caso l'induttore è costituito da due avvolgimenti concentrici
e coassiali, ciascuno dotato di 1500 spire, che montati in serie hanno un
fattore di auto-induzione $L \sim 0.5 \; \si{\henry}$ e i condensatori hanno
capacità dai valori nominali: $C = \left\{0.1,\; 0.22,\; 0.47\right\} \pm
10\% \; \si{\micro\farad}$. La componente resistiva $R$ del circuito è data
dalle resistenze interne del generatore di tensione ($r_G = 50 \; \si{\ohm}$
nominali) e dell'induttore, che indichiamo con: $r \approx 40 \;\si{\ohm}$.
Le $3000$ spire totali di filo di rame negli avvolgimenti infatti
influiscono apprezzabilmente e in maniera non banale sul trasferimento
di energia all'interno del circuito.

\begin{figure}[!htp]
	\centering 
		\includegraphics[width=14cm]{RLCschm}
	\caption{Diagramma del circuito RLC studiato\label{schm: RLC}}
\end{figure}

\subsubsection{Oscillatore a reazione con BJT}
L'anello di feedback soddisfa il criterio di stabilità di Barkhausen,
lo sfasamento totale

\begin{figure}[!htp]
	\centering 
		\includegraphics[width=14cm]{BJTschm}
	\caption{Schema circuitale dell'oscillatore a reazione con BJT
	studiato\label{schm: BJT}}
\end{figure}
%================================================================
%                   Analisi dati e Risultati
%================================================================
\section{Analisi dati e Risultati}
Per poter condurre un'analisi sui dati raccolti \`e stato innanzitutto
necessario convertire le letture digitalizzate nelle corrispondenti
grandezze fisiche, le coppie tensione-corrente relative al diodo.
Inizialmente si sono convertite le acquisizioni e le incertezze associate in
d.d.p. tramite i fattori di conversione per entrambi gli ADC,
determinati come descritto nel paragrafo \ref{sec: cal}.
Dunque, dalla caduta di tensione ai capi della resistenza $R1$ possiamo
determinare la corrente di lavoro del diodo grazie alla legge di Ohm. 
Si \`e quindi effettuato un filtraggio volto all'eliminazione degli outliers e 
dei punti meno significativi, assumendoli quali variabili indipendenti e di 
natura gaussiana. 

\subsection{Oscillatore a reazione con BJT}

\iffalse
Si riportano i valori ottimali dei parametri stimati dal fit e le relative
covarianze: 
\begin{align*}
R_{\text{diodo}} &= 46.112\iffalse 18 \fi \pm 0.007\iffalse 2 \fi \; \si{\mohm} 
&\sigma_{I_0, \eta V_T} &= 0.97  \\
\eta V_T &= 47.579\iffalse 86 \fi \pm 0.003\iffalse 3 \fi \; \si{\mV} 	
&\sigma_{I_0, R_d} &= -0.62 \\
I_0 &= 4.518\iffalse 0 \fi \pm 0.004\iffalse 3 \fi \; \si{\nA}
&\sigma_{I_0, \text{ofst}} &= -0.49 \\
\text{offset} &=\; - 2.204\iffalse 3 \fi \pm 0.007 \iffalse 3 \fi \; \si{\uA}
&\sigma_{\eta V_T, R_d} &= -0.70  \\ 
\chi^2/\text{ndof} &= 72207/251066
&\sigma_{\eta V_T, \text{ofst}} &= -0.43\\
\text{abs\_sigma} &= \rm False
&\sigma_{R_d, \text{ofst}} &= 0.22
\end{align*}
Infine si mostrano i dati acquisiti con sovrapposta la funzione di best-fit
nei grafici \ref{fig: sck_lin}, in scala lineare, e \ref{fig: sck_log}
in scala semilogaritmica. 

\begin{table}[H]
\begin{center}
\begin{tabular}{lll}
	\toprule
	$R1$ nom. [$\si{\ohm}$] & $R1$ mis. [$\si{\ohm}$] \\ 
	\midrule
	\midrule
	$0.22 \pm 3 \% $         	& $0.226 \pm 0.008$ \\
	$2.2 \pm 5 \% $          	& $2.212 \pm 0.008$ \\
	$22 \pm 5 \% $           	& $21.86 \pm 0.010$ \\ 
	$220 \pm 5 \% $          	& $216.22 \pm 0.07$ \\
	$2.2\; \si{k} \pm 5 \% $       & $2202.1 \pm 0.4$ \\
	$22\; \si{k} \pm 5 \% $       & ($21.7 \pm 0.3)10^3$ \\
	$0.22\; \si{M} \pm 5 \% $      & ($217 \pm 3)10^3$ \\
	\bottomrule
\end{tabular}
\caption{I valori delle resistenze poste in serie al diodo, riportate in
	valore nominale e misurate con multimetro digitale. \label{tab: res}}
\end{center}
\end{table}
\fi
Per l'implementazione si rimanda allo
\href{https://github.com/BernardoTomellier/FFT/tree/master/fft_plot.py}
{script}, dove \verb+lab.py+ fornisce le funzioni di appoggio e
\verb+data.py+ definisce i parametri fondamentali in ingresso.

\begin{figure}[!ht]
\centering
	\includegraphics{phs1_41.pdf}
\label{fig: phs1_41}
\end{figure}
Si noti come, alla fine del segnale oscillante\footnote{alla fine del
semiperiodo positivo dell'onda quadra in uscita dal generatore di funzioni}
si riesce ad apprezzare un picco positivo di circa $100$ digit $\approx 97$ mV,
questo coincide esattamente con il fronte di salita dell'onda quadra in
ingresso al circuito RLC.
Questo si deve alla somma di due effetti: Quando l'onda passa da LOW a
HIGH, la corrente che scorre nella maglia RLC passa da circa $-100$ a
$100 \si{\micro\A}$. Dunque la bobina da $L=0.5 \si{\H}$ fornisce una tensione
$DV(t) = L \frac{\ud I}{\ud t}$ secondo la legge degli induttori.
Un secondo possibile contributo al picco di tensione alla
fine del segnale oscillante è dovuto all'accoppiamento capacitivo/capacità
parassita del diodo, che possiamo modellare come $Cjo = 1-4 \si{\pico\farad}$
in parallelo al diodo. Quando il fronte d'onda sale rapidamente, parte dell'onda
quadra viene lasciata passare dal condensatore, aggiungendo qualche $\si{\nano\V}$
al picco finale.
\iffalse
\begin{figure}[H]
	\centering 
		\includegraphics[width=16cm, height= 11cm]
		{100skip_linear}
	\caption{Dati acquisiti e funzione di best-fit \eqref{eq: model}. E' 
	stato rappresentato un punto ogni 100 per comodit\`a di visualizzazione.
	\label{fig: sck_lin}}
\end{figure}

\begin{figure}[!htp]
	\centering 
		\includegraphics[width=16cm, height= 9.9cm]{10skip_semilog}
	\caption{Dati acquisiti e funzione di best fit \eqref{eq: model} in 
	scala semilogaritmica. A scopo illustrativo sono stati rappresentati anche
	i dati della serie $220\si{\kohm}$. E' stato disegnato un punto ogni 10
	per comodit\`a di visualizzazione. \label{fig: sck_log}}
\end{figure}
\fi

\subsection{Nota sull'implementazione}
Per determinare i parametri ottimali e le rispettive covarianze si \`e
implementato in \verb+Python+ un algoritmo di fit basato sui minimi quadrati
mediante la funzione \emph{curve\_fit} della libreria 
\texttt{SciPy}\cite{scipy}. Il modulo \texttt{signal} della stessa libreria
è stato usato per l'implementazione dei filtri numerici e delle finestre,
mentre l'implementazione dell'algoritmo per il calcolo della FFT è lo stesso
definito dalla libreria \texttt{NumPy}\cite{numpy}
Per tutti i fit su campionamenti digitali di \verb+Arduino+ si \`e imposto
$\rm{abs\_sigma} =$ True, avendo preso come incertezza associata il valore
convenzionale $\sigma = 1$ digit, per cui effettivamente si sta eseguendo
un fit dei minimi quadrati. Per tutti gli altri campionamenti, in cui la
sorgente principale di incertezza abbia natura non statistica o non meglio
determinata si è posto $\rm{abs\_sigma} =$ False. 
%================================================================
%                          Conclusioni
%================================================================
\section{Conclusioni}

%================================
%        Filtro outliers
%================================
\subsection{Filtro outliers}
La parte pi\`u semplice nel filtraggio dati consiste nello scartare tutti quei
punti che distano da $\mu_y$ pi\`u di una soglia arbitraria $k$ di deviazioni
standard $\sigma_y$ (nel nostro caso \`e stato scelto $k = 3$).
Si rimanda ai 
\href{https://github.com/BernardoTomelleri/FFT/tree/master}{sorgenti}.
\iffalse
\begin{figure}[!htbp]
\centering
\begin{subfigure}{.5\textwidth}
	\centering 
 		\includegraphics[scale=0.5]{./nofilter.png}
		\caption{\label{fig: nofilter}}	
\end{subfigure}%
\begin{subfigure}{.5\textwidth}
	\centering 
 		\includegraphics[scale=0.5]{./filtered.png}
 	\caption{\label{fig: filtered}}
\end{subfigure}
\caption{Grafici in scala semilogaritmica prima (\ref{fig: nofilter}) e dopo} 
\end{figure}
\fi


%================================================================
%                            END
%================================================================
\medskip
\bibliographystyle{IEEEtrandoi}
\bibliography{refs}
\end{document}
